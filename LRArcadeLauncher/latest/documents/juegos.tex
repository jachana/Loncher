\documentclass[language=spanish]{article}
\usepackage{fancyhdr}
\usepackage[utf8]{inputenc}
\pagestyle{fancy}
\usepackage[margin=0.5in]{geometry}
\usepackage{booktabs}
\lhead{Arcade La Resistencia\\Documento especificación de juegos\\Versión 1.0 - \today \\}
\begin{document}

\title{Especificación de juegos Arcade 1.0}
\author{Jurgen Heysen}
\date{\today}
\maketitle
\newpage

\tableofcontents
\newpage

\section{Introducción}

El presente documento tiene como objetvo instruir a los desarrolladores de videojuegos sobre los estándares que deben cumplir para ser compatibles con el Launcher del Aracde

\section{Estándar}

\subsection{Código}

Los juegos deben exponer un archivo .py con una clase que contenga el método {\tt Go}, que debe recibir como parámetro un objeto del tipo {\tt ArcadeServiceInterface} (ArcadeServiceInterface.py), este objeto será tratado mas adelante y es una interfaz a los servicios solicitados por el juego.

\subsection{XML}

Los juegos deben exponer un archivo .xml con información relevante sobre ellos.
El nodo principal debe llamarse \textless Game\textgreater y opcionalmente de argumento puede llevar un string llamado version con el valor "1.0".
El Titulo del juego va entre las etiquetas \textless Title\textgreater y \textless /Title\textgreater. Es importante agregar un código corto único para el juego entre las etiquetas \textless Code\textgreater y \textless /Code\textgreater.\\
Las etiquetas \textless Description\textgreater y \textless /Description\textgreater encierran la descripción del juego.\\
La versión se incluye entre \textless Version\textgreater y \textless /Version\textgreater , se recomienda que sea un entero\\
La fecha de compilación puede ser añadida con \textless Date\textgreater  y \textless /Date\textgreater \\
La etiqueta \textless MainClass..\textgreater  es obligatoria y debe llevar como argumentos name con el nombre de la clase que contiene el método Go(...) y path con la ruta relativa a este archivo, extensión incluida.\\
Se puede incluir la lista de autores mediante una sección \textless Authors\textgreater  finalizada con \textless /Authors\textgreater . Cada autor se individualiza \textless Author name="..."/\textgreater , dónde name es el nombre del autor.\\
Otro aspecto a considerar es la lista contenida entre \textless Screenshots\textgreater  y \textless /Screenshots\textgreater  para incluir las screenshots del juego que se mostraran en el Launcher. cada una se indica con la etiqueta \textless Screenshot src="..."/\textgreater  donde src es el path a la screenshot.\\
La lista denotada por \textless Services\textgreater y \textless /Services\textgreater  indica los servicios que solicita el juego al Launcher, cada servicio se solicita mediante \textless Service name="..." .../\textgreater  donde name es el nombre del servicio solicitado, además se deben incluir los parámetros requeridos por el servicio según su propio estándar.\\
Adicionalmente se puede crear una lista denotada por \textless AdditionalData\textgreater  y \textless /AdditionalData\textgreater  para incluir datos adicionales. No hay especificaciones sobre los contenidos de esta lista.\\ 
A continuación se presenta una tabla resumen sobre la información que debe contener el xml:

% Table generated by Excel2LaTeX from sheet 'Hoja1'
\begin{table}[htbp]
  \centering
  \caption{Resumen de documento Xml}
    \begin{tabular}{rrrrr}
    \toprule
    \multicolumn{1}{c}{\textbf{Tag}} & \multicolumn{1}{c}{\textbf{Parent}} & \multicolumn{1}{c}{\textbf{Args}} & \multicolumn{1}{c}{\textbf{Num}} & \multicolumn{1}{c}{\textbf{Descripción}} \\
    \midrule
    Game  & -     & -     & 1     & Root del documento \\
    Title & Game  & -     & 1     & Título del juego \\
    Code  & Game  & -     & 1     & Código del juego \\
    MainClass & Game  & name: Nombre & 1     & Nombre de la clase de  \\
          &       & file: archivo &       & inicialización del juego \\
    Description & Game  & -     & 1     & Descripción \\
    Version & Game  & -     & 1     & Versión del juego \\
    Date  & Game  & -     & 1     & Fecha de release \\
    Authors & Game  & -     & 1     & Encierra lista de Authores \\
    Author & Authors & name: Nombre & *     & Identifica un autor \\
    Screenshots & Game  & -     & 1     & Encierra lista de capturas de pantalla \\
    Screenshot & Screenshots & src: path a la imagen & *     & Identifica una captura de pantalla \\
    Services & Game  & -     & 1     & Encierra lista de servicios \\
    Service & Services & name: Nombre del servicio & *     & Identifica un servicio que solicita el juego \\
          &       & parámetros adicionales del servicio &       &  \\
    AdditionalData & Game  &       & 1     & Encierra una lista de datos adicionales \\
    \bottomrule
    \end{tabular}%
  \label{tab:tablaxmll}%
\end{table}%


\section{Servicios}

El Launcher ofrece a los juegos diversos servicios que pueden ser instalados. Estos son agegados a la instalación base y siguiendo la documentación de servicios, los desarrolladores pueden incluir los suyos propios.\\
En esta sección se abordará la interfaz común de acceso a servicios.

\subsection{Clase ArcadeServiceInterface}

La clase {\tt ArcadeServiceInterface} provee el acceso a los servicios del Arcade que son declarados en el .xml de información del juego. Los servicios contenidos en él se encuentran ya inicializados para el juego, se provee dos formas de acceder a los servicios.

\subsubsection{Método getServices()}

Este método devuelve un diccionario con las clases que representan cada servicio, donde las llaves son los nombres de estos. Si faltara algún servicio, signiica que este no se registró correctamente en la inicialización o que no se encuentra instalado, pero para todos los efectos no está disponible.

\subsubsection{Método getService(servicio)}

Este método recibe como string el nombre del servicio y retorna  el objeto que representa a este servicio si se encuentra disponible o None en caso contrario.

\section{Instalación de juegos}

En una distribución normal del Launcher, los juegos pueden ser instalados siguiendo los siguientes pasos:\\
\begin{enumerate}
	\item Descomprimir los contenidos del juego en una carpeta de elección.
	\item Navegar a carpeta del Launcher.
	\item En consola de comandos, ejecutar {\tt python game\_install.py -i path\_a\_xml} \footnote{Ajustar a python 2.7 si no fuera el caso}
	\item Verificar con {\tt python game\_install.py -l}
\end{enumerate}
Puede desinstalar\footnote{No borra los archivos, solo quita el juego del registro del arcade} un juego con {\tt python game\_install.py -u CODIGO} donde CODIGO es el codigo del juego que desea desinstalar. Puede ser visto ejecutando antes {\tt python game\_install.py -l}

\end{document}